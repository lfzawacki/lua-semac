\documentclass[a4paper,12pt]{abnt}

\usepackage[utf8]{inputenc}
\usepackage[brazil]{babel}
\usepackage{listings}
\usepackage{verbatim}
\usepackage[usenames,dvipsnames]{color}

%opening
\titulo{A Linguagem Lua}
\comentario{Anos perdidos na confusa imensidão da Galáxia, enfim,
elegante refúgio sob a luz da Lua}
\autor{Lucas Fialho Zawacki \\ Luca Couto Manique Barreto}

%-------------------------------------------------------------------

\begin{document}

%-------------------------------------------------------------------

\folhaderosto
\sumario

\input{./lua.def}
\lstset{language=Lua,
	extendedchars=true,
	inputencoding=utf8,
	basicstyle=\ttfamily,
   keywordstyle=\color{BlueViolet},
   commentstyle=\color{BrickRed},
   stringstyle=\color{ForestGreen},
  % numbers=left,
   numberstyle=\tiny\color{RedViolet},
  % stepnumber=1,
   numbersep=10pt,
   backgroundcolor=\color{white},
   tabsize=2,
   showspaces=false,
	showstringspaces=false,
	texcl=true,
	escapechar={\©},
	breaklines=true
}

\chapter{Introdução}

Criada em 1993 pelo Grupo de Tecnologia em Computação Gráfica (TecGraf) da
Pontíficie Universidade Católica do Rio de Janeiro (PUC/Rio) \cite{luaabout},
Lua é uma linguagem de script simples, porém poderosa.
Tendo todo seu código implementado em pouco mais de
17 mil linhas (e pesando menos de 300KB) \cite{luaabout}, esta linguagem
tira suas forças do C, já que foi feita com o propósito de ser embarcada
em aplicações já existentes.

A grosso modo, Lua é um conjunto de bibliotecas ANSI C que provêm
funções para rodar scripts, chamar funções C de dentro
do código Lua (e vice-versa), manipular parâmetros da máquina virtual
e trocar dados entre os dois ambientes.

\section{Motivação}

A linguagem foi escolhida por causa do nicho que preenche:
aplicações que depedendem da eficiência e disponibilidade da linguagem C,
mas que necessitam de uma maneira fácil, rápida e indolor de extensão de
funcionalidades. Este nicho engloba game engines \cite{luascript},
aplicações em hardwares embarcados, servidores web e outros.
Porém a linguagem evoluiu muito desde sua criação e já
demonstrou potencial em diversas áreas, como programação científica \cite{luacient},
especificação de dados e (recentemente) até frameworks
para desenvolvimento web \cite{luaweb} !

Algumas features que despertaram o interesse:

\begin{itemize}
\item Linguagem de Script
\item Integração com C (Rápida)
\item OpenSource
\item JIT Compiling na máquina virtual (MAIS Rápida!)
\item Coleta de Lixo mark-and-sweep incremental (bom para aplicações real-time)
\item Funções como valores de primeira ordem.
\item Expressões Regulares (muito próximas de POSIX)
\item Facilidade de criar meta-linguagens e especificar dados

\end{itemize}

\section{Instalação}

Pelo seu caráter embarcado, muitas vezes o interpretador da linguagem já virá
"instalado" dentro do ambiente que a utilizará. Mesmo assim, podemos querer
instalar as versões \emph{stand-alone} do interpretador ou até obter o código
fonte para podermos embarcá-la em nossas próprias aplicações. Distribuições da
linguagem podem ser encontradas em \cite{luaorg}.

\chapter{Sintaxe}

A sintaxe é minimalista e visa usar mais palavras do que símbolos.
Um exemplo seria a delimitação de novos blocos de código é
feita com as palavras \emph{do end}. Outro é o uso de vários operadores como
palavras, em oposição a símbolos, como  \emph{and}, \emph{or} e \emph{not}.

\section{Palavras reservadas}

Estas são palavras reservadas pelo interpretador da linguagem e não podem ser
utilizadas como nomes de variáveis.

\begin{lstlisting}
and       break     do        else      elseif
end       false     for       function  if
in        local     nil       not       or
repeat    return    then      true      until
while
\end{lstlisting}

\section{Operadores}

A seguir uma tabela dos operadores da linguagem, em ordem de precedência:

\begin{lstlisting}
^
not  - (unario)
*   /
+   -
..
<   >   <=  >=  ~=  ==
and
or
\end{lstlisting}

\section{Declaração de Variáveis}

Em Lua existem três tipos de variáveis podem ser declaradas:
globais, locais e campos de tabelas. A diferença entre
elas é a visibilidade.

Variáveis passam a existir à partir da próxima linha depois de sua definição.
Caso nada seja especificado na hora da definição
a variável será considerada global,
para criar variáves locais é preciso prefixar a definição
com a palavra reservada \emph{local}.
Seguem alguns exemplos de declaração de variáveis:

\begin{lstlisting}
-- definindo a e b como variaveis locais
local a = 2
local b = "yeah"

-- c e uma variavel global
c = 10

-- uma tabela
local d = {}

-- nome e um campo da tabela b
d.nome = b
print(d.nome)
\end{lstlisting}

Uma variável sempre contém o valor \emph{nil} antes de ser inicializada,
isto serve também para campos de uma tabela e é útil para testar
se variáveis estão ou não definidas.

\chapter{Escopos}

\section{Visibilidade e Acesso de Variáveis}

Construindo à partir do que foi falado no último capítulo,
existem dois tipos de escopo em Lua: local e global.

\begin{lstlisting}
-- variavel disponiveis em todos os escopos
a = 1

-- variavel local
local b = 2

function foo() do
    print(a)
    local a = 2
    print(a + b)
end
\end{lstlisting}

Como era de se esperar variáveis globais podem ser acessadas em qualquer
\emph{chunk} (mais sobre eles à seguir) a qualquer momento.
Note que a funcao teve acesso à variável \emph{a}, naturalmente,
mas também conseguiu sem problemas manipular
a variável \emph{b} que não pertencia ao seu escopo.

Isto acontece porque em Lua os escopos mais aninhados tem acesso às
variáveis locais dos escopos superiores.
Estas variáveis são chamadas up-values e podem ser usadas para
implementar um conceito de linguagens
funcionais conhecido como \emph{closures} \cite{closures}.

\section{Chunks}

Todo código que roda na VM do Lua é tratado como um chunk \cite{pil}.
Ele pode ser uma linha (o interpretador recebendo texto e executando),
ou qualquer número de comandos concatenados. O interpretador Lua sempre
cria um chunk à partir do código que está sendo lido no momento e
portanto o "binário" executado é simplesmente uma sequência de chunks.

Como benefício desta abordagem temos que é possível misturar código Lua
pré-compilado com scripts de maneira transparente.

\chapter{Controle de fluxo}

Não podiam faltar na linguagem os comandos para delinear o
fluxo de controle do programa, aqui é demonstrada a
sintaxe de cada um lado a lado com exemplos.


\section{if then else}

\begin{lstlisting}
if exp then
	block
{elseif exp then
	block }
[else block]
end

local e = 10
local num = 2^e
if num == 512 then
	print("e = " .. 9)
elseif num == 1024 then
	print("e = " .. 10)
else
	print("Sei la entao...")
end
\end{lstlisting}

\section{while}

\begin{lstlisting}
while e do
	block
end

-- euclides
local a = 21
local b = 3
while a ~= b do
	if a > b then
		a = a - b
	else
		b = b - a
	end
end
print('MDC: ' .. a)
\end{lstlisting}

\section{for numérico}

\begin{lstlisting}
for Name = e1 , e2 [, e3] do
	block
end

local numbers = { 0 , 99, 3, 20 , 51 }

for i = 1,table.getn(numbers) do
	print(numbers[i])
end
\end{lstlisting}

\section{for genérico}

\begin{lstlisting}
for namelist in explist do
	block
end


local food = { "banana", "alface", "batata", "batata-frita" }

for i in pairs(food) do
	print('food[' .. i .. '] tem: ' .. food[i])
end

for k,v in ipairs(food) do
	print('food[' .. k .. '] tem: ' .. v)
end
\end{lstlisting}

\chapter{Tipos}

\section{Number}

Number é o nome do tipo usado em Lua para representar números. Eles
são equivalentes ao tipo \emph{double} da linguagem C, o que
significa que ela não possui um tipo inteiro nativo.
Caso seja necessário é possível compilar a linguagem para usar
como Number um \emph{int}, \emph{long int} ou algum outro tipo.

\section{Nil}

Tipo que possui apenas um valor \emph{nil} e equivalente a void em C/C++.
Em Lua, atribuir nil a uma variável significa deletá-la.

\section{Boolean}

Além do padrão \emph{true}/\emph{false}, expressões em Lua podem ter seu
"valor-verdade" avaliado de outra forma: Para a linguagem,
qualquer tipo diferente de \emph{nil} e \emph{false}
(inclusive o número 0, o char '\textbackslash0' e afins)
é considerado \emph{true}.

\section{Strings}

As strings são um tipo simples da linguagem e tem algumas
características interessantes.
Toda string é um objeto único na memória, isto quer dizer que cada
interpretador Lua só tem em sua memória uma cópia de
cada string usada pelo programa, mesmo que existam várias
ocorrências dela sendo usadas.

Além disso elas são imutáveis - não é possível modificá-las,
apenas retornar uma nova com a
modificação - fazendo com que cada string seja
mapeada para um simples endereço de memória.

Algumas das consequências disso são comparações de
string em tempo \emph{O(1)}, maior \emph{overhead} para criação de strings
e menor uso de memória em casos onde existem muitas ocorrências repetidas.
Além disso uma clara desvantagem é a perda de eficiência em situações
que trabalham pesadamente fazendo modifições na mesma string,
já que elas nunca serão feitas \emph{in place} e decorrerá a criação de várias
cópias intermediárias \cite{pil}.

\begin{lstlisting}
--	exemplo de como nao escrever codigo com strings
--	a criacao constante de novos valores pode ficar pesada
-- WARNING: bad code ahead!!
local buff = ""
for line in io.lines() do
    buff = buff .. line .. "\n"
end
\end{lstlisting}

Como ficará claro mais adiante, a comparação de strings é usada
pesadamente nos acessos a tabelas e portanto o trade-off
entre mutabilidade e imutabilidade faz bastante sentido.

\subsection{Biblioteca Padrão}

Na biblioteca padrão dentro da tabela \emph{string}
temos várias funções para manipular strings.

\begin{lstlisting}
local str = 'nova string'
local outra_str = "com aspas duplas"
print( str .. ' com ' .. outra_str)
print( string.gsub(str , 'nova' , 'velha') )

-- saudades do printf?
local idade = 19
local nome = 'luke'
print(string.format('Oi eu sou %s e tenho %d anos.\n',nome,idade) )
\end{lstlisting}

\section{Funções}

Em Lua funções são valores de primeira ordem,
atente para os seguintes exemplos:

\begin{lstlisting}
-- declaracao oldschool de funcao
function foo(x)
    return x*x
end

-- equivalente a declaração acima, a variavel
-- referencia uma funcao anonima
goo = function (x)
    return x*x
end

print( foo(2) + goo(3) )
\end{lstlisting}

Como podemos ver, uma função é apenas um valor que é associado a um
nome usando-se uma variável. Por esse motivo, operações como a seguinte
são possíveis.

\begin{lstlisting}
-- um pouco de magica
a = print

print = function (...)
   a('Nao, funcao errada cara')
end

print('Algo')
print('Por favor')
a('Agora sim!')

\end{lstlisting}

Rodar o exemplo acima pode ser esclarecedor, mas basicamente o que se vê aqui
é que os nomes de funções não são amarrados estaticamente aos
valores que carregam, são simples variáveis.
s
Além disso, funções podem ser associadas à variáveis locais, o que possibilita
o seguinte código:

\begin{lstlisting}
do -- do end delimitam um novo escopo
	local a = print
	print = function (...) -- uma funcao anonima que recebe 'n' parametros
		a('Compre nossos produtos!')
		a(...) -- imprime os parametros passados para a funcao
		a('Esta propaganda e um oferecimento das closures!')
	end
end

print("Ok","vamos ver")
print(2+10)
a("Vamos tentar...") -- a não e deste escopo

\end{lstlisting}

Outra feature interessante é a possibilidade de
múltiplos retornos em uma função \cite{pil} :

\begin{lstlisting}
-- multiplos retornos
function maximo (a)

  local mi = 1          -- indice do maximo
  local m = a[mi]       -- valor do maximo

  for i,val in ipairs(a) do
    if val > m then
      mi = i
      m = val
    end
  end

  return m, mi
end

print( maximo({3,99,12,50}) )

\end{lstlisting}

\section{Tabelas}

Além dos tipos primitivos em Lua existe um outro tipo muito especial: a tabela.
Uma tabela é um \emph{array associativo} que usa um hash de mapeamento \cite{pil}
para relacionar chaves com valores.
É uma estrutura de dados extremamente simples e versátil e,
impressionantemente, é a única que está presente em Lua!

\subsection{Vetores e Estruturas}

Não existem arrays (vetores) e structs (estruturas), os dois configuram
casos especiais de uma tabela,
alguns exemplos de tabela são os seguintes:

\begin{lstlisting}
-- declaracao de uma tabela vazia
a = {}
-- preenchendo com valores
a["lucas"] = 10
a["jesus"] = 0
a[1] = 3

-- exemplo de array
array = { 4, 3, 2 , 1 }
print(a[1]) -- 4
print(a[3]) -- 2

-- exemplo de struct
peep = {}
peep.name = "lucas" -- equivalente a peep["name"] = "lucas"
peep["age"] = 19 -- usando a notação original, também podia ser peep.age = 19
print(peep.name .. " | " .. peep.age)
\end{lstlisting}

\subsection{Namespaces}

Além disso tabelas podem ser usadas para emular o conceito \emph{namespace}
presente em outras linguagens. A biblioteca padrão, por exemplo, está definida
dentro de diversas tabelas (math, os, string, ...)

\begin{lstlisting}
-- namespace 'fun'
local fun = {}

-- função definida no campo foo da tabela
fun.foo = function (x,y)
    return (x+y)^2
end

-- açucar sintático
function fun.goo (a,b)
    return a^2 + 2*a*b + b^2
end

print(fun.foo(3,4))
print(fun.goo(5,2))
\end{lstlisting}

\subsection{Objetos}

Como já visto em 5.6.1, tabelas podem ser tratadas como estruturas, com os campos
de tipos numérico, booleano e string representando atributos.
Além disso, graças ao fato de Lua tratar funções como variáveis de primeira
ordem, estas também podem ser associadas a campos de tabelas, o que lembra o
conceito de métodos, permitindo tratar tabelas como objetos e nos aproximando
do paradigma de Programação Orientada a Objeto (POO, vista adiante).

\subsection{Metatables}

Como o próprio nome sugere, Metatables são tabelas que possuem métodos, chamados
metamethods, que descrevem o comportamento de outras tabelas.
Existem metamethods de operações aritméticas (como \_\_add), relacionais
(por exemplo, \_\_eq), entre outros (como \_\_index, método responsável por
descrever o comportamento de uma tabela quando um campo não inicializado é chamado).

\subsection{Orientação a Objetos}

A relação entre Metatables e Tables parece bastante análoga à relação Classe-Objeto.
Lua não possui mecanismos próprios da linguagem para Programação Orientada a Objeto,
entretanto, possui diversos mecanismos que possibilitam 'simular' este paradigma.
De fato, o que existe em Lua é \emph{POO} Baseada em Protótipos \cite{pbo} , onde a herança de
classes se dá a partir de um objeto da classe pai que passa a exercer o papel de classe filha.

Outro recurso disponível em Lua para facilitar a POO é o açucar sintático
objeto:metodo() (equivalente a \emph{objeto.metodo(objeto)} ), que elimina a
necessidade de passar o objeto que invoca o método como parâmetro explicitamente.

Como podemos ver, as tabelas são um jeito bastante elegante e uniforme de
representar diversas construções.
Como dito anteriormente elas são o único tipo composto disponível e
diversas abstrações podem ser representadas à partir delas.

\section{Outros tipos}

Além dos tipos citados anteriormente existem ainda mais dois:
\emph{Userdata} e \emph{Thread}. O primeiro representa um bloco de dados qualquer que
foi recebido do ``{}lado C''{} do programa (tipicamente um ponteiro para uma região de memória).
O segundo é usado para implementar concorrência de uma maneira bem simples.
Nenhum dos dois tipos será discutido em mais profundidade aqui. Para maiores
informações \cite{pil} e \cite{luamanual} podem ser consultados.

\chapter{Garbage Collection}

Garbage Collection (ou coleta de lixo) é o nome da técnica usada por Lua, e
por várias outras linguagens, para aliviar do programador a responsabilidade
de gerenciar a alocação e desalocação de memória, bem como ficar fazendo
malabarismos com as referências para a memória que lhe ``{}pertence''{}.

Isso facilita um estilo de programação mais relaxado,
criando e descartando valores
intermediários de forma indiscriminada, o que pode se tornar um problema de performance,
principalmente em nichos como o desenvolvimento de
jogos que exigem respostas em tempo real.
Felizmente para aqueles que precisam da eficiência o
interpretador provê artifícios para controlar o
coletor de lixo, para que ele rode mais espaçadamente,
exatamente quando for pedido ou até mesmo pare por completo.
Além disso, o algoritmo utilizado é o mark-n-sweep incremental \cite{luamanual}
que é bastante eficiente para aplicações que dependem de um tempo de resposta rápido.

\chapter{Integração com C API}

Grande parte do charme da linguagem vem do fato
que ela é construida em cima da base sólida da linguagem C.
Embora hoje Lua já seja bastante usada como uma
linguagem \emph{stand-alone} nos mesmos
moldes que Python, Perl e Ruby a ``{}vocação''{} principal dela é
ser usada embarcada em outras aplicações.
Isto é feito de maneira bem simples e direta: incluindo o
código fonte do interpretador Lua juntamente
com o código do programa e usando uma API simples para
fazer a comunicação entre os dois lados.

Utilizando esta abordagem é possível, por exemplo:

\begin{itemize}%
\item Rodar scripts Lua de dentro do seu programa
\item Rodar scripts Lua toda vez que algum evento ocorre no seu programa (hooks)
\item Expor dados do seu programa para os scripts Lua (e vice-versa)
\item Expor funções escritas em código C puro para scripts Lua (bindings)

\end{itemize}

\section{Rodando scripts}

A seguir um exemplo de um programa C que roda scripts Lua,
primeiramente simplificado para melhor ilustrar o conceito:

\begin{lstlisting}[language=C]
#include <stdio.h>
#include <stdlib.h>

//a "complicacao" esta aqui dentro deste arquivo
#include "lua-open.h"

int main(int argc, char** argv)
{
    ©//©um lua_State* e um interpretador e seu estado interno
    lua_State *L = openLua();

    if (argc > 1) {
	    printf("Hello from C\n");
	    //uma funcao para carregar um arquivo lua e executa-lo
	    loadLuaFile(L,argv[1]);
    } else {
	    printf("Give me a lua script as an argument!\n ./" __FILE__ " something.lua\n");
    }

}\end{lstlisting}

Os arquivos \emph{lua-open.h} e \emph{lua-open.c} contém o resto do código:

\begin{lstlisting}[language=C]
//open-lua.h
//interpretador lua e a API
#include <lua.h>
#include <lauxlib.h>
#include <lualib.h>

lua_State* openLua();
void loadLuaFile(lua_State* L, const char *filename);

#include "lua-open.h"

lua_State* openLua()
{
    ©//©inicializando lua_State* novo
    lua_State *L = lua_open();

    ©//©todas as funcoes da API trabalham com um lua_State*
    ©//©como primeiro parametro

    //bibliotecas padrao do lua sendo abertas
    luaopen_base(L);
    luaopen_table(L);
    luaopen_string(L);
    luaopen_math(L);

    return L;

}

//equivalente ao loadfile() usado de dentro do lua
void loadLuaFile(lua_State* L, const char *filename)
{
    //carrega o arquivo e compila num chunk
    int error = luaL_loadfile(L, filename);

    //reporta o erro
    if(error)
      printf("Error opening file: %s\n",lua_tostring(L, -1));

    //efetivamente executa o código carregado
    lua_pcall(L, 0, 0, 0);
}
\end{lstlisting}

\section{Pilha}

Cada interpretador Lua possui uma
pilha de parâmetros que estão sendo manipulados no momento e o
programa C pode manipular estes valores para se comunicar com os scripts
Lua.

\section{Mais detalhes}

É possível fazer muito mais com a C API do que o
demonstrado aqui, mas isto foge ao escopo deste trabalho.
Como sempre, maiores informações podem ser encontradas em \cite{luamanual} e
\cite{pil}.

\chapter{Conclusão}

Neste trabalho, fizemos uma passada rápida por vários aspectos da linguagem de
programação Lua. Este estudo só comprovou as hipóteses que tínhamos de que
a linguagem é muito interessante e promisora. Ficamos muito impressionados
com a elegência e poder dela quando opostos à sua simplicidade, velocidade e
tamanho. Com certeza ela é uma das melhores escolhas para para desenvolver
aplicações que tenham uma forte dependência na velocidade de execução sem
poderem deixar de lado o poder de expressão, a facilidade de programação e
a extensibilidade.

\bibliography{referencias}
\bibliographystyle{abnt-num}

\end{document}

